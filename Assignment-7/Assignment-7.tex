\documentclass[journal,12pt,twocolumn]{IEEEtran}

\usepackage{setspace}
\usepackage{gensymb}
\singlespacing
\usepackage[cmex10]{amsmath}

\usepackage{amsthm}

\usepackage{mathrsfs}
\usepackage{txfonts}
\usepackage{stfloats}
\usepackage{bm}
\usepackage{cite}
\usepackage{cases}
\usepackage{subfig}

\usepackage{longtable}
\usepackage{multirow}
\usepackage{caption}

\usepackage{enumitem}
\usepackage{mathtools}
\usepackage{steinmetz}
\usepackage{tikz}
\usepackage{circuitikz}
\usepackage{verbatim}
\usepackage{tfrupee}
\usepackage[breaklinks=true]{hyperref}
\usepackage{graphicx}
\usepackage{tkz-euclide}
\usepackage{float}

\usetikzlibrary{calc,math}
\usepackage{listings}
    \usepackage{color}                                            %%
    \usepackage{array}                                            %%
    \usepackage{longtable}                                        %%
    \usepackage{calc}                                             %%
    \usepackage{multirow}                                         %%
    \usepackage{hhline}                                           %%
    \usepackage{ifthen}                                           %%
    \usepackage{lscape}     
\usepackage{multicol}
\usepackage{chngcntr}

\DeclareMathOperator*{\Res}{Res}

\renewcommand\thesection{\arabic{section}}
\renewcommand\thesubsection{\thesection.\arabic{subsection}}
\renewcommand\thesubsubsection{\thesubsection.\arabic{subsubsection}}

\renewcommand\thesectiondis{\arabic{section}}
\renewcommand\thesubsectiondis{\thesectiondis.\arabic{subsection}}
\renewcommand\thesubsubsectiondis{\thesubsectiondis.\arabic{subsubsection}}


\hyphenation{op-tical net-works semi-conduc-tor}
\def\inputGnumericTable{}                                 %%

\lstset{
%language=C,
frame=single, 
breaklines=true,
columns=fullflexible
}
\begin{document}

\newcommand{\BEQA}{\begin{eqnarray}}
\newcommand{\EEQA}{\end{eqnarray}}
\newcommand{\define}{\stackrel{\triangle}{=}}
\bibliographystyle{IEEEtran}
\raggedbottom
\setlength{\parindent}{0pt}
\providecommand{\mbf}{\mathbf}
\providecommand{\pr}[1]{\ensuremath{\Pr\left(#1\right)}}
\providecommand{\qfunc}[1]{\ensuremath{Q\left(#1\right)}}
\providecommand{\sbrak}[1]{\ensuremath{{}\left[#1\right]}}
\providecommand{\lsbrak}[1]{\ensuremath{{}\left[#1\right.}}
\providecommand{\rsbrak}[1]{\ensuremath{{}\left.#1\right]}}
\providecommand{\brak}[1]{\ensuremath{\left(#1\right)}}
\providecommand{\lbrak}[1]{\ensuremath{\left(#1\right.}}
\providecommand{\rbrak}[1]{\ensuremath{\left.#1\right)}}
\providecommand{\cbrak}[1]{\ensuremath{\left\{#1\right\}}}
\providecommand{\lcbrak}[1]{\ensuremath{\left\{#1\right.}}
\providecommand{\rcbrak}[1]{\ensuremath{\left.#1\right\}}}
\theoremstyle{remark}
\newtheorem{rem}{Remark}
\newcommand{\sgn}{\mathop{\mathrm{sgn}}}
\providecommand{\abs}[1]{\vert#1\vert}
\providecommand{\res}[1]{\Res\displaylimits_{#1}} 
\providecommand{\norm}[1]{\lVert#1\rVert}
%\providecommand{\norm}[1]{\lVert#1\rVert}
\providecommand{\mtx}[1]{\mathbf{#1}}
\providecommand{\mean}[1]{E[ #1 ]}
\providecommand{\fourier}{\overset{\mathcal{F}}{ \rightleftharpoons}}
%\providecommand{\hilbert}{\overset{\mathcal{H}}{ \rightleftharpoons}}
\providecommand{\system}{\overset{\mathcal{H}}{ \longleftrightarrow}}
	%\newcommand{\solution}[2]{\textbf{Solution:}{#1}}
\newcommand{\solution}{\noindent \textbf{Solution: }}
\newcommand{\cosec}{\,\text{cosec}\,}
\providecommand{\dec}[2]{\ensuremath{\overset{#1}{\underset{#2}{\gtrless}}}}
\newcommand{\myvec}[1]{\ensuremath{\begin{pmatrix}#1\end{pmatrix}}}
\newcommand{\mydet}[1]{\ensuremath{\begin{vmatrix}#1\end{vmatrix}}}
\numberwithin{equation}{subsection}
\makeatletter
\@addtoreset{figure}{problem}
\makeatother
\let\StandardTheFigure\thefigure
\let\vec\mathbf
\renewcommand{\thefigure}{\theproblem}
\def\putbox#1#2#3{\makebox[0in][l]{\makebox[#1][l]{}\raisebox{\baselineskip}[0in][0in]{\raisebox{#2}[0in][0in]{#3}}}}
     \def\rightbox#1{\makebox[0in][r]{#1}}
     \def\centbox#1{\makebox[0in]{#1}}
     \def\topbox#1{\raisebox{-\baselineskip}[0in][0in]{#1}}
     \def\midbox#1{\raisebox{-0.5\baselineskip}[0in][0in]{#1}}
\vspace{3cm}

\title{AI1103-Assignment-7}
\author{Name: Vikhyath Sai Kothamasu, Roll Number: CS20BTECH11056}
\maketitle
\newpage
\bigskip
\renewcommand{\thefigure}{\theenumi}
\renewcommand{\thetable}{\theenumi}

\begin{figure} [h]
    \includegraphics[width = 0.3\textwidth]{college logo.png}
\end{figure}

Download all python codes from 
\begin{lstlisting}
https://github.com/Vikhyath-vec/AI1103/tree/main/Assignment-7/codes
\end{lstlisting}
%
and latex-tikz codes from 
%
\begin{lstlisting}

https://github.com/Vikhyath-vec/AI1103/blob/main/Assignment-7/Assignment-7.tex
\end{lstlisting}
\section*{Question}
A fair die is thrown two times independently. Let $X,Y$ be the outcomes of these two throws and $Z=X+Y$. Let $U$ be the remainder obtained when $Z$ is divided by 6. Then which of the following statement(s) is/are true?
\begin{enumerate}
    \item $X$ and $Z$ are independent \label{option 1}
    \item $X$ and $U$ are independent \label{option 2}
    \item $Z$ and $U$ are independent \label{option 3}
    \item $Y$ and $Z$ are not independent \label{option 4}
\end{enumerate}

\section*{Solution}

Let $X \in \{1,2,3,4,5,6\}$ represent the random variable which represents the outcome of the first throw of a dice. Similarly, $Y \in \{1,2,3,4,5,6\}$ represents the random variable which represents the outcome of the second throw of a dice.
\begin{align}
    n(X=i) = 1, \quad i \in \{1, 2, 3, 4, 5, 6\}
\end{align}

\begin{align}
    \Pr(X=i) = 
	\begin{cases}
	\frac{1}{6}   &  i \in \{1, 2, 3, 4, 5, 6\}\\ ~\\[-1em]
	0 & \text{otherwise}
	\end{cases}
\end{align}
Similarly, 
\begin{align}
    \Pr(Y=i) = 
	\begin{cases}
	\frac{1}{6}   &  i \in \{1, 2, 3, 4, 5, 6\}\\ ~\\[-1em]
	0 & \text{otherwise}
	\end{cases}
\end{align}
\begin{align}
    Z &= X+Y
    \\ \text{Let } z &\in \{1, 2, \hdots, 11, 12\}
    \\\Pr{(Z=z)} &= \Pr{(X+Y = z)}
    \\ &= \sum_{x=0}^z \Pr{(X=x)}\Pr{(Y=z-x)}
    \\ &= (6 - \abs{z-7}) \times\frac{1}{6}\times\frac{1}{6}
    \\ &= \frac{6 - \abs{z-7}}{36}
    \\\Pr(Z=z) &= 
	\begin{cases}
	\frac{6 - \abs{z-7}}{36}   &  z \in \{1, 2, \hdots, 11, 12\}\\ ~\\[-1em]
	0 & \text{otherwise}
	\end{cases}
\end{align}
$U$ is the remainder obtained when $Z$ is divided by 6.
\begin{align}
    \text{Let } u &\in \{0, 1, 2, 3, 4, 5\}
    \\\Pr{(U=u)} &= \sum_{k=0}^2\Pr{(Z = 6k+u)}
    \\\Pr{(U=0)} &= \Pr{(Z = 0)} + \Pr{(Z = 6)} + \Pr{(Z = 12)}
    \\ &= 0 + \frac{5}{36} + \frac{1}{36} = \frac{1}{6}
    \\ \text{for } u &\in \{1, 2, 3, 4, 5\}
    \\\Pr{(U=u)} &= \Pr{(Z = 0+u)} + \Pr{(Z = 6+u)}
    \\&= \frac{6 - \abs{u-7}}{36} +  \frac{6 - \abs{6+u-7}}{36}
    \\&= \frac{6 - (7-u)}{36} +  \frac{6 - (u-1)}{36}
    \\&= \frac{u - 1 + 7 - u}{36} = \frac{6}{36}
    \\&=\frac{1}{6}
    \\\Pr(U=u) &= 
	\begin{cases}
    \frac{1}{6}   &  u \in \{0, 1, 2, 3, 4, 5\}\\ ~\\[-1em]
	0 & \text{otherwise}
	\end{cases}
\end{align}
Now, for checking each option,
\begin{enumerate}
    
\item Checking if $X$ and $Z$ are independent
\begin{align}
    p_1 &= \Pr{(Z=z, X=x)}
    \\ &= \Pr{(Y=z-x, X=x)}
    \\ &= \Pr{(Y=z-x)} \times \Pr{(X=x)}
    \\ &= \begin{cases}
        \frac{1}{36} & z-x \in \{1, 2, 3, 4, 5, 6\}\\ ~\\[-1em]
        0 & \text{otherwise}
    \end{cases}
\end{align}
\begin{align}
    \Pr{(Z=z)}\times \Pr{(X=x)} &= \frac{6 - \abs{z-7}}{36} \times \frac{1}{6}
    \\&= \frac{6 - \abs{z-7}}{216}
    \\\Pr{(Z=z)}\Pr{(X=x)} &\neq \Pr{(Z=z, X=x)}  \label{equation 1}
\end{align}
$X$ and $Z$ are not independent from \eqref{equation 1} and hence option \eqref{option 1} is false.

\item Checking if $X$ and $U$ are independent
\begin{align}
    p_2 = \Pr{(U=u, X=x)}
\end{align}
\begin{multline}
    p_2 = \Pr{((Z=u) + (Z=6+u)}
    \\+ (Z=12+u), X=x)
\end{multline}
\begin{multline}
    p_2 = \Pr{((Y=u-x) + (Y=6+u-x)}
    \\+ (Y=12+u-x), X=x)
\end{multline}
\begin{align}
    p_2 &= \frac{1}{6} \times \frac{1}{6}
    \\&= \frac{1}{36}
\end{align}
\begin{align}
    \Pr{(U=u)}\times \Pr{(X=x)} &= \frac{1}{6} \times \frac{1}{6}
    \\&= \frac{1}{36}
    \\\Pr{(U=u)}\Pr{(X=x)} &= \Pr{(U=u, X=x)}  \label{equation 2}
\end{align}
$X$ and $U$ are independent from \eqref{equation 2} and hence option \eqref{option 2} is true.
\item Checking if $Z$ and $U$ are independent
\begin{align}
    p_3 &= \Pr{(Z=z| U=u)}
    \\p_3 &= 
    \begin{cases}
        1 & u=1 \text{ and } z=7\\ ~\\[-1em]
        \frac{1}{2} & u=0 \text{ and } z\in\{6,12\}\\ ~\\[-1em]
        \frac{1}{2} & u\in\{2,3,4,5\}  \text{ and } \\&z=u \text{ or } z=6+u\\ ~\\[-1em]
        0 & \text{otherwise}
    \end{cases}
    \\\Pr{(Z=z)} &= \frac{6 - \abs{z-7}}{36}
\end{align}
If $Z$ and $U$ are independent, then
\begin{align}
    \Pr{(Z=z| U=u)} &= \frac{\Pr{(Z=z, U=u)}}{\Pr{(U=u)}}
    \\&= \frac{\Pr{(Z=z)}\Pr{(U=u)}}{\Pr{(U=u)}}
    \\&= \Pr{(Z=z)}
\end{align}
But,
\begin{align}
    \Pr{(Z=z| U=u)} \neq \Pr{(Z=z)} \label{equation 3}
\end{align}
$X$ and $U$ are not independent from \eqref{equation 3} and hence option \eqref{option 3} is false.
\item Checking if $Y$ and $Z$ are independent
\begin{align}
    p_1 &= \Pr{(Z=z, Y=y)}
    \\ &= \Pr{(X=z-y, Y=y)}
    \\ &= \Pr{(X=z-y)} \times \Pr{(Y=y)}
    \\ &= \begin{cases}
        \frac{1}{36} & z-y \in \{1, 2, 3, 4, 5, 6\}\\ ~\\[-1em]
        0 & \text{otherwise}
    \end{cases}
\end{align}
\begin{align}
    \Pr{(Z=z)}\times \Pr{(Y=y)} &= \frac{6 - \abs{z-7}}{36} \times \frac{1}{6}
    \\&= \frac{6 - \abs{z-7}}{216}
    \\\Pr{(Z=z)}\Pr{(Y=y)} &\neq \Pr{(Z=z, Y=y)}  \label{equation 4}
\end{align}
$X$ and $Z$ are not independent from \eqref{equation 4} and hence option \eqref{option 4} is true.
\end{enumerate}
Thus, options \eqref{option 2} and \eqref{option 4} are true.
\end{document}