\documentclass[journal,12pt,twocolumn]{IEEEtran}

\usepackage{setspace}
\usepackage{gensymb}
\singlespacing
\usepackage[cmex10]{amsmath}

\usepackage{amsthm}

\usepackage{mathrsfs}
\usepackage{txfonts}
\usepackage{stfloats}
\usepackage{bm}
\usepackage{cite}
\usepackage{cases}
\usepackage{subfig}

\usepackage{longtable}
\usepackage{multirow}
\usepackage{caption}

\usepackage{enumitem}
\usepackage{mathtools}
\usepackage{steinmetz}
\usepackage{tikz}
\usepackage{circuitikz}
\usepackage{verbatim}
\usepackage{tfrupee}
\usepackage[breaklinks=true]{hyperref}
\usepackage{graphicx}
\usepackage{tkz-euclide}

\usetikzlibrary{calc,math}
\usepackage{listings}
    \usepackage{color}                                            %%
    \usepackage{array}                                            %%
    \usepackage{longtable}                                        %%
    \usepackage{calc}                                             %%
    \usepackage{multirow}                                         %%
    \usepackage{hhline}                                           %%
    \usepackage{ifthen}                                           %%
    \usepackage{lscape}     
\usepackage{multicol}
\usepackage{chngcntr}

\DeclareMathOperator*{\Res}{Res}

\renewcommand\thesection{\arabic{section}}
\renewcommand\thesubsection{\thesection.\arabic{subsection}}
\renewcommand\thesubsubsection{\thesubsection.\arabic{subsubsection}}

\renewcommand\thesectiondis{\arabic{section}}
\renewcommand\thesubsectiondis{\thesectiondis.\arabic{subsection}}
\renewcommand\thesubsubsectiondis{\thesubsectiondis.\arabic{subsubsection}}


\hyphenation{op-tical net-works semi-conduc-tor}
\def\inputGnumericTable{}                                 %%

\lstset{
%language=C,
frame=single, 
breaklines=true,
columns=fullflexible
}
\begin{document}

\newcommand{\BEQA}{\begin{eqnarray}}
\newcommand{\EEQA}{\end{eqnarray}}
\newcommand{\define}{\stackrel{\triangle}{=}}
\bibliographystyle{IEEEtran}
\raggedbottom
\setlength{\parindent}{0pt}
\providecommand{\mbf}{\mathbf}
\providecommand{\pr}[1]{\ensuremath{\Pr\left(#1\right)}}
\providecommand{\qfunc}[1]{\ensuremath{Q\left(#1\right)}}
\providecommand{\sbrak}[1]{\ensuremath{{}\left[#1\right]}}
\providecommand{\lsbrak}[1]{\ensuremath{{}\left[#1\right.}}
\providecommand{\rsbrak}[1]{\ensuremath{{}\left.#1\right]}}
\providecommand{\brak}[1]{\ensuremath{\left(#1\right)}}
\providecommand{\lbrak}[1]{\ensuremath{\left(#1\right.}}
\providecommand{\rbrak}[1]{\ensuremath{\left.#1\right)}}
\providecommand{\cbrak}[1]{\ensuremath{\left\{#1\right\}}}
\providecommand{\lcbrak}[1]{\ensuremath{\left\{#1\right.}}
\providecommand{\rcbrak}[1]{\ensuremath{\left.#1\right\}}}
\theoremstyle{remark}
\newtheorem{rem}{Remark}
\newcommand{\sgn}{\mathop{\mathrm{sgn}}}
\providecommand{\abs}[1]{\vert#1\vert}
\providecommand{\res}[1]{\Res\displaylimits_{#1}} 
\providecommand{\norm}[1]{\lVert#1\rVert}
%\providecommand{\norm}[1]{\lVert#1\rVert}
\providecommand{\mtx}[1]{\mathbf{#1}}
\providecommand{\mean}[1]{E[ #1 ]}
\providecommand{\fourier}{\overset{\mathcal{F}}{ \rightleftharpoons}}
%\providecommand{\hilbert}{\overset{\mathcal{H}}{ \rightleftharpoons}}
\providecommand{\system}{\overset{\mathcal{H}}{ \longleftrightarrow}}
	%\newcommand{\solution}[2]{\textbf{Solution:}{#1}}
\newcommand{\solution}{\noindent \textbf{Solution: }}
\newcommand{\cosec}{\,\text{cosec}\,}
\providecommand{\dec}[2]{\ensuremath{\overset{#1}{\underset{#2}{\gtrless}}}}
\newcommand{\myvec}[1]{\ensuremath{\begin{pmatrix}#1\end{pmatrix}}}
\newcommand{\mydet}[1]{\ensuremath{\begin{vmatrix}#1\end{vmatrix}}}
\numberwithin{equation}{subsection}
\makeatletter
\@addtoreset{figure}{problem}
\makeatother
\let\StandardTheFigure\thefigure
\let\vec\mathbf
\renewcommand{\thefigure}{\theproblem}
\def\putbox#1#2#3{\makebox[0in][l]{\makebox[#1][l]{}\raisebox{\baselineskip}[0in][0in]{\raisebox{#2}[0in][0in]{#3}}}}
     \def\rightbox#1{\makebox[0in][r]{#1}}
     \def\centbox#1{\makebox[0in]{#1}}
     \def\topbox#1{\raisebox{-\baselineskip}[0in][0in]{#1}}
     \def\midbox#1{\raisebox{-0.5\baselineskip}[0in][0in]{#1}}
\vspace{3cm}
\title{AI1103-Assignment-2}
\author{Name: Vikhyath Sai Kothamasu, Roll Number: CS20BTECH11056}
\maketitle
\newpage
\bigskip
\renewcommand{\thefigure}{\theenumi}
\renewcommand{\thetable}{\theenumi}
Download all python codes from 
\begin{lstlisting}
https://github.com/Vikhyath-vec/AI1103/tree/main/Assignment-2/codes
\end{lstlisting}
%
and latex-tikz codes from 
%
\begin{lstlisting}

https://github.com/Vikhyath-vec/AI1103/blob/main/Assignment-2/Assignment-2.tex
\end{lstlisting}
\section*{Question}
Two cards are drawn successively with replacement from a well shuffled deck of 52 cards. Find the probability distribution of the number of aces.

\section*{Solution}

Let $X \in \{0,1,2,3,4,5,6,7,8,9,10,11,12\}$ represent the random variable, where 0 represents an ace card, 1 represents a card numbered '2', 2 represents a card numbered '3'... 9 represents a card numbered '10', 10 represents the J card, 11 represents the Q card, and 12 represents the K card. These are independent of the suit. From the given information, we know that there exists 4 different card of each number in a well-shuffled deck of 52 cards. Thus
\begin{align}
    n(X=i) = 4, i \in \{0, 1, 2\hdots 10, 11, 12\}
\end{align}
Since the deck is well shuffled and complete. the probability of finding a card of a given number is:
\begin{align}
    \Pr(X=i) = 
	\begin{cases}
	\dfrac{4}{52} = \dfrac{1}{13} &  i \in \{0, 1, 2\hdots 10, 11, 12\}\\ ~\\[-1em]
	0 & \text{otherwise}
	\end{cases}
\end{align}
Since we are picking only 2 cards, the number of aces which can be obtained is either 0 or 1 or 2. Let Y $\in \{0,1,2\}$ represent the random variable, where 0 represents the case where no aces are selected, 1 represents the case where one ace is selected, 2 represents the case where 2 aces are selected.
\\Let Z $\in \{0,1\}$ represent the random variable, where 0 represents an ace card is picked while 1 represents a non-ace card is picked.
From equation (0.0.2), probability of selecting an ace card is:
\begin{align}
    \Pr(Z=0) = \frac{1}{13}
\end{align}
Similarly, probability of selecting a non-ace card is:
\begin{align}
    \Pr(Z=1) = \Sigma_{i=1}^{12} Pr(X=i)
    \\ = \Sigma_{i=1}^{12}\frac{1}{13}
    \\ = \frac{12}{13}
\end{align}
Now, for finding the probability distribution of the number of aces, $\Pr(Y=0)$ would mean 0 aces are selected or 2 non-ace cards are selected. Thus,
\begin{align}
    \Pr(Y=0) = \frac{12}{13} \times \frac{12}{13} = \frac{144}{169}
    \\\Pr(Y=0) = 0.852071
\end{align}
$\Pr(Y=1)$ would mean 1 ace and 1 non-ace card are selected. This can be done in 2 ways namely: 
\begin{enumerate}
    \item first selecting an ace card and then selecting a non-ace card 
    \item first selecting a non-ace card and then selecting an ace card
\end{enumerate}
Thus,
\begin{align}
    \Pr(Y=1) = \frac{1}{13} \times \frac{12}{13} + \frac{12}{13} \times \frac{1}{13} = \frac{24}{169}
    \\\Pr(Y=1) = 0.142012
\end{align}
$\Pr(Y=2)$ would mean 2 aces are selected. Thus,
\begin{align}
    \Pr(Y=2) = \frac{1}{13} \times \frac{1}{13} = \frac{1}{169}
    \\\Pr(Y=2) = 0.005917
\end{align}
\begin{center}
\begin{tabular}{ | m{2.5cm} | m{2cm}| m{4cm} | } 
\hline
Serial number & Case & Probability of the case \\ 
\hline
1 & $\Pr(Y=0)$ & 0.852071 \\ 
\hline
2 & $\Pr(Y=1)$ & 0.142012 \\ 
\hline
3 & $\Pr(Y=2)$ & 0.005917 \\
\hline

\end{tabular}
\end{center}
Above is the probability distribution table of the number of aces obtained when two cards are drawn successively with replacement from a well shuffled deck of 52 cards.
\\Below is the graph with theoretical and simulated result of the probability distribution.
\begin{figure}[h]
    \includegraphics[width = 0.5\textwidth]{assignment2 graph.png}
    \caption{theoretical and simulated probability results}
\end{figure}
\end{document}