\documentclass[journal,12pt,twocolumn]{IEEEtran}

\usepackage{setspace}
\usepackage{gensymb}
\singlespacing
\usepackage[cmex10]{amsmath}

\usepackage{amsthm}

\usepackage{mathrsfs}
\usepackage{txfonts}
\usepackage{stfloats}
\usepackage{bm}
\usepackage{cite}
\usepackage{cases}
\usepackage{subfig}

\usepackage{longtable}
\usepackage{multirow}
\usepackage{caption}

\usepackage{enumitem}
\usepackage{mathtools}
\usepackage{steinmetz}
\usepackage{tikz}
\usepackage{circuitikz}
\usepackage{verbatim}
\usepackage{tfrupee}
\usepackage[breaklinks=true]{hyperref}
\usepackage{graphicx}
\usepackage{tkz-euclide}
\usepackage{float}

\usetikzlibrary{calc,math}
\usepackage{listings}
    \usepackage{color}                                            %%
    \usepackage{array}                                            %%
    \usepackage{longtable}                                        %%
    \usepackage{calc}                                             %%
    \usepackage{multirow}                                         %%
    \usepackage{hhline}                                           %%
    \usepackage{ifthen}                                           %%
    \usepackage{lscape}     
\usepackage{multicol}
\usepackage{chngcntr}

\DeclareMathOperator*{\Res}{Res}

\renewcommand\thesection{\arabic{section}}
\renewcommand\thesubsection{\thesection.\arabic{subsection}}
\renewcommand\thesubsubsection{\thesubsection.\arabic{subsubsection}}

\renewcommand\thesectiondis{\arabic{section}}
\renewcommand\thesubsectiondis{\thesectiondis.\arabic{subsection}}
\renewcommand\thesubsubsectiondis{\thesubsectiondis.\arabic{subsubsection}}


\hyphenation{op-tical net-works semi-conduc-tor}
\def\inputGnumericTable{}                                 %%

\lstset{
%language=C,
frame=single, 
breaklines=true,
columns=fullflexible
}
\begin{document}

\newcommand{\BEQA}{\begin{eqnarray}}
\newcommand{\EEQA}{\end{eqnarray}}
\newcommand{\define}{\stackrel{\triangle}{=}}
\bibliographystyle{IEEEtran}
\raggedbottom
\setlength{\parindent}{0pt}
\providecommand{\mbf}{\mathbf}
\providecommand{\pr}[1]{\ensuremath{\Pr\left(#1\right)}}
\providecommand{\qfunc}[1]{\ensuremath{Q\left(#1\right)}}
\providecommand{\sbrak}[1]{\ensuremath{{}\left[#1\right]}}
\providecommand{\lsbrak}[1]{\ensuremath{{}\left[#1\right.}}
\providecommand{\rsbrak}[1]{\ensuremath{{}\left.#1\right]}}
\providecommand{\brak}[1]{\ensuremath{\left(#1\right)}}
\providecommand{\lbrak}[1]{\ensuremath{\left(#1\right.}}
\providecommand{\rbrak}[1]{\ensuremath{\left.#1\right)}}
\providecommand{\cbrak}[1]{\ensuremath{\left\{#1\right\}}}
\providecommand{\lcbrak}[1]{\ensuremath{\left\{#1\right.}}
\providecommand{\rcbrak}[1]{\ensuremath{\left.#1\right\}}}
\theoremstyle{remark}
\newtheorem{rem}{Remark}
\newcommand{\sgn}{\mathop{\mathrm{sgn}}}
\providecommand{\abs}[1]{\vert#1\vert}
\providecommand{\res}[1]{\Res\displaylimits_{#1}} 
\providecommand{\norm}[1]{\lVert#1\rVert}
%\providecommand{\norm}[1]{\lVert#1\rVert}
\providecommand{\mtx}[1]{\mathbf{#1}}
\providecommand{\mean}[1]{E[ #1 ]}
\providecommand{\fourier}{\overset{\mathcal{F}}{ \rightleftharpoons}}
%\providecommand{\hilbert}{\overset{\mathcal{H}}{ \rightleftharpoons}}
\providecommand{\system}{\overset{\mathcal{H}}{ \longleftrightarrow}}
	%\newcommand{\solution}[2]{\textbf{Solution:}{#1}}
\newcommand{\solution}{\noindent \textbf{Solution: }}
\newcommand{\cosec}{\,\text{cosec}\,}
\providecommand{\dec}[2]{\ensuremath{\overset{#1}{\underset{#2}{\gtrless}}}}
\newcommand{\myvec}[1]{\ensuremath{\begin{pmatrix}#1\end{pmatrix}}}
\newcommand{\mydet}[1]{\ensuremath{\begin{vmatrix}#1\end{vmatrix}}}
\numberwithin{equation}{subsection}
\makeatletter
\@addtoreset{figure}{problem}
\makeatother
\let\StandardTheFigure\thefigure
\let\vec\mathbf
\renewcommand{\thefigure}{\theproblem}
\def\putbox#1#2#3{\makebox[0in][l]{\makebox[#1][l]{}\raisebox{\baselineskip}[0in][0in]{\raisebox{#2}[0in][0in]{#3}}}}
     \def\rightbox#1{\makebox[0in][r]{#1}}
     \def\centbox#1{\makebox[0in]{#1}}
     \def\topbox#1{\raisebox{-\baselineskip}[0in][0in]{#1}}
     \def\midbox#1{\raisebox{-0.5\baselineskip}[0in][0in]{#1}}
\vspace{3cm}

\title{AI1103-Assignment-8}
\author{Name: Vikhyath Sai Kothamasu, Roll Number: CS20BTECH11056}
\maketitle
\newpage
\bigskip
\renewcommand{\thefigure}{\theenumi}
\renewcommand{\thetable}{\theenumi}

\begin{figure} [h]
    \includegraphics[width = 0.3\textwidth]{college logo.png}
\end{figure}


%
Download all latex-tikz codes from 
%
\begin{lstlisting}

https://github.com/Vikhyath-vec/AI1103/blob/main/Assignment-8/Assignment-8.tex
\end{lstlisting}
\section*{Question}
Suppose $X_1$ and $X_2$ are independent and identically distributed random variables each following an exponential distribution with mean $\theta$, i.e., the common pdf is given by $f_\theta(x) = \frac{1}{\theta}e^{\frac{-x}{\theta}}, 0<x<\infty,0<\theta<\infty.$ Then which of the following is true? Conditional distribution of $X_2$ given $X_1+X_2=t$ is 
\begin{enumerate}
    \item exponential with mean $\frac{t}{2}$ and hence $X_1+X_2$ is sufficient for $\theta$ \label{option 1}
    \item exponential with mean $\frac{t\theta}{2}$ and hence $X_1+X_2$ is not sufficient for $\theta$ \label{option 2}
    \item uniform$(0,t)$ and hence $X_1+X_2$ is sufficient for $\theta$ \label{option 3}
    \item uniform$(0,t\theta)$ and hence $X_1+X_2$ is not sufficient for $\theta$ \label{option 4}
\end{enumerate}
    
\section*{Solution}
Let $f_{X_1,X_2}(x_1,x_2)$ denote the joint probability distribution of random variables $X_1$ and $X_2$. Let $Z$ be another random variable such that $Z=X_1+X_2$. Let $\Phi_{X_1}(\omega)$ and $\Phi_{Z}(\omega)$ be the characteristic functions of the probability density $f_{X_1}(x)$ and $f_{Z}(x)$ respectively. Also, given in the question,
\begin{align}
    0 &< \theta < \infty
    \\f_{X_1}(x_1) &= \frac{1}{\theta}e^{\frac{-x_1}{\theta}}, 0<x_1<\infty
    \\f_{X_2}(x_2) &= \frac{1}{\theta}e^{\frac{-x_2}{\theta}}, 0<x_2<\infty
\end{align}
Since $X_1$ and $X_2$ are independent, 
\begin{align}
f_{X_1,X_2}(x_1,x_2) &= f_{X_1}(x_1) \times f_{X_2}(x_2)
    \\&= \frac{1}{\theta}e^{\frac{-x_1}{\theta}} \times \frac{1}{\theta}e^{\frac{-x_2}{\theta}}
    \\&= \frac{1}{\theta^2}e^{\frac{-(x_1+x_2)}{\theta}}
    \\\Phi_{X_1}(\omega) &= \frac{1}{\theta} \int_{0}^{\infty}e^{i\omega x} e^{\frac{-x}{\theta}} \,dx
    \\ &= \frac{1}{\theta} \times \frac{1}{i\omega - \frac{1}{\theta}} \brak{e^{x(i\omega - \frac{1}{\theta}}}\bigg\vert_0^{\infty}
    \\ &= \frac{1}{1-i\omega\theta} - \frac{\lim_{x\rightarrow \infty} \brak{e^{x(i\omega - \frac{1}{\theta}}}}{1-i\omega\theta}
    \\&= \frac{1}{1-i\omega\theta} - 0 = \frac{1}{1-i\omega\theta} 
    \\ \Phi_{Z}(\omega) &= \brak{\frac{1}{1-i\omega\theta} }^2
    \\ f_Z(x) &= \frac{1}{2\pi} \int_{-\infty}^{\infty}\frac{e^{-i\omega x}}{\brak{\frac{1}{1-i\omega\theta} }^2} \,d\omega \label{equation 1}
\end{align}
The equation \eqref{equation 1} is the characteristic function expression of a gamma random variable with k=2. Thus,
\begin{align}
    f_Z(x) &= \frac{x^{k-1}e^{\frac{-x}{\theta}}}{\Gamma(k)\theta^k}
    \\ &=  \frac{x^{2-1}e^{\frac{-x}{\theta}}}{\Gamma(2)\theta^2}
    \\ &= \frac{xe^{\frac{-x}{\theta}}}{\theta^2}
\end{align}
\begin{align}
    f_{X_2|(X_1+X_2=t)}(x_2) = 
    \begin{cases}
    \frac{f_{X_1,X_2}(x_1,x_2)}{f_Z(t)} &  x_2 \in [0, t]\\ ~\\[-1em]
    0 & \text{otherwise}
    \end{cases}
\end{align}
Let $ x_2 \in [0, t]$.
\begin{align}
    f_{X_2|(X_1+X_2=t)}(x_2) &= \frac{f_{X_1,X_2}(x_1,x_2)}{f_Z(t)}
    \\&= \frac{\frac{1}{\theta^2}e^{\frac{-(x_1+x_2)}{\theta}}}{\frac{1}{\theta^2}e^{\frac{-t}{\theta}}t}
    \\&= \frac{e^{\frac{-(t)}{\theta}}}{e^{\frac{-t}{\theta}}t}
    \\&= \frac{1}{t} \quad \forall x_2 \in [0, t]
\end{align}
The obtained pdf is uniform$(0,t)$. And since the conditional distribution of $X_2$ does not depend on $\theta$ for any value of $t$, $X_1+X_2$ is sufficient for $\theta$. Verifying the pdf,
\begin{align}
    \text{total probability} &= \int_{0}^{t} f_{X_2|(X_1+X_2=t)}(x_2) \,dx_2
    \\&= \int_{0}^{t} \frac{1}{t} \,dx_2
    \\&= 1
\end{align}
Hence, the correct answer is option \eqref{option 3}
\end{document}