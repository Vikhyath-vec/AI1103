\documentclass[journal,12pt,twocolumn]{IEEEtran}

\usepackage{setspace}
\usepackage{gensymb}
\singlespacing
\usepackage[cmex10]{amsmath}

\usepackage{amsthm}
\usepackage{amssymb}
\usepackage{mathrsfs}
\usepackage{txfonts}
\usepackage{stfloats}
\usepackage{bm}
\usepackage{cite}
\usepackage{cases}
\usepackage{subfig}

\usepackage{longtable}
\usepackage{multirow}
\usepackage{caption}

\usepackage{enumitem}
\usepackage{mathtools}
\usepackage{steinmetz}
\usepackage{tikz}
\usepackage{circuitikz}
\usepackage{verbatim}
\usepackage{tfrupee}
\usepackage[breaklinks=true]{hyperref}
\usepackage{graphicx}
\usepackage{tkz-euclide}
\usepackage{float}

\usetikzlibrary{calc,math}
\usepackage{listings}
    \usepackage{color}                                            %%
    \usepackage{array}                                            %%
    \usepackage{longtable}                                        %%
    \usepackage{calc}                                             %%
    \usepackage{multirow}                                         %%
    \usepackage{hhline}                                           %%
    \usepackage{ifthen}                                           %%
    \usepackage{lscape}     
\usepackage{multicol}
\usepackage{chngcntr}

\DeclareMathOperator*{\Res}{Res}

\renewcommand\thesection{\arabic{section}}
\renewcommand\thesubsection{\thesection.\arabic{subsection}}
\renewcommand\thesubsubsection{\thesubsection.\arabic{subsubsection}}

\renewcommand\thesectiondis{\arabic{section}}
\renewcommand\thesubsectiondis{\thesectiondis.\arabic{subsection}}
\renewcommand\thesubsubsectiondis{\thesubsectiondis.\arabic{subsubsection}}


\hyphenation{op-tical net-works semi-conduc-tor}
\def\inputGnumericTable{}                                 %%

\lstset{
%language=C,
frame=single, 
breaklines=true,
columns=fullflexible
}
\begin{document}

\newcommand{\BEQA}{\begin{eqnarray}}
\newcommand{\EEQA}{\end{eqnarray}}
\newcommand{\define}{\stackrel{\triangle}{=}}
\bibliographystyle{IEEEtran}
\raggedbottom
\setlength{\parindent}{0pt}
\providecommand{\mbf}{\mathbf}
\providecommand{\pr}[1]{\ensuremath{\Pr\left(#1\right)}}
\providecommand{\qfunc}[1]{\ensuremath{Q\left(#1\right)}}
\providecommand{\sbrak}[1]{\ensuremath{{}\left[#1\right]}}
\providecommand{\lsbrak}[1]{\ensuremath{{}\left[#1\right.}}
\providecommand{\rsbrak}[1]{\ensuremath{{}\left.#1\right]}}
\providecommand{\brak}[1]{\ensuremath{\left(#1\right)}}
\providecommand{\lbrak}[1]{\ensuremath{\left(#1\right.}}
\providecommand{\rbrak}[1]{\ensuremath{\left.#1\right)}}
\providecommand{\cbrak}[1]{\ensuremath{\left\{#1\right\}}}
\providecommand{\lcbrak}[1]{\ensuremath{\left\{#1\right.}}
\providecommand{\rcbrak}[1]{\ensuremath{\left.#1\right\}}}
\theoremstyle{remark}
\newtheorem{rem}{Remark}
\newcommand{\sgn}{\mathop{\mathrm{sgn}}}
\providecommand{\abs}[1]{\vert#1\vert}
\providecommand{\res}[1]{\Res\displaylimits_{#1}} 
\providecommand{\norm}[1]{\lVert#1\rVert}
%\providecommand{\norm}[1]{\lVert#1\rVert}
\providecommand{\mtx}[1]{\mathbf{#1}}
\providecommand{\mean}[1]{E[ #1 ]}
\providecommand{\fourier}{\overset{\mathcal{F}}{ \rightleftharpoons}}
%\providecommand{\hilbert}{\overset{\mathcal{H}}{ \rightleftharpoons}}
\providecommand{\system}{\overset{\mathcal{H}}{ \longleftrightarrow}}
	%\newcommand{\solution}[2]{\textbf{Solution:}{#1}}
\newcommand{\solution}{\noindent \textbf{Solution: }}
\newcommand{\cosec}{\,\text{cosec}\,}
\providecommand{\dec}[2]{\ensuremath{\overset{#1}{\underset{#2}{\gtrless}}}}
\newcommand{\myvec}[1]{\ensuremath{\begin{pmatrix}#1\end{pmatrix}}}
\newcommand{\mydet}[1]{\ensuremath{\begin{vmatrix}#1\end{vmatrix}}}
\numberwithin{equation}{subsection}
\makeatletter
\@addtoreset{figure}{problem}
\makeatother
\let\StandardTheFigure\thefigure
\let\vec\mathbf
\renewcommand{\thefigure}{\theproblem}
\def\putbox#1#2#3{\makebox[0in][l]{\makebox[#1][l]{}\raisebox{\baselineskip}[0in][0in]{\raisebox{#2}[0in][0in]{#3}}}}
     \def\rightbox#1{\makebox[0in][r]{#1}}
     \def\centbox#1{\makebox[0in]{#1}}
     \def\topbox#1{\raisebox{-\baselineskip}[0in][0in]{#1}}
     \def\midbox#1{\raisebox{-0.5\baselineskip}[0in][0in]{#1}}
\vspace{3cm}

\title{Challenging Problem 13}
\author{Name: Vikhyath Sai Kothamasu, Roll Number: CS20BTECH11056}
\maketitle
\newpage
\bigskip
\renewcommand{\thefigure}{\theenumi}
\renewcommand{\thetable}{\theenumi}

\begin{figure} [h]
    \includegraphics[width = 0.3\textwidth]{college logo.png}
\end{figure}


%
Download all python codes from 
\begin{lstlisting}
https://github.com/Vikhyath-vec/AI1103/tree/main/Challenging-Problem-13/codes
\end{lstlisting}
%
and latex-tikz codes from 
%
\begin{lstlisting}

https://github.com/Vikhyath-vec/AI1103/blob/main/Challenging-Problem-13/Challenging-Problem-13.tex
\end{lstlisting}
\section*{Question}
If each element of a n order determinant is either zero or one, what is the probability that the value of the determinant is positive? (Assume that the individual entries are chosen independently each value being assumed with probability $\frac{1}{2}$)
    
\section*{Solution}
Let $X \in \{0,1\}$ be a random variable denoting the value of each entry. Let $Y \in \{0, ,1, 2\}$ be the random variable denoting the sign of determinant, i.e., 0 denotes a negative determinant, 1 denotes a zero determinant (singular matrix), and 2 denotes a positive determinant.
\begin{align}
    \Pr{(X=0)} = \frac{1}{2}
    \\ \Pr{(X=1)} = \frac{1}{2}
\end{align}

Here, field of order 2 is $\mathbb{F}_2 = \mathbb{Z}_2 = {0, 1}$. The probability of determinant not being zero (non-singular matrices) is:
\begin{align}
    = \Pr{(Y=0)} + \Pr{(Y=2)}
\end{align}
For a matrix to be non-singular, we have to make sure that all the rows are linearly independent and non-zero. Let us consider each row of the matrix to be a vector. A set of vectors is said to be linearly dependent if there is a nontrivial linear combination of the vectors that equals the zero vector. If no such linear combination exists, then the vectors are said to be linearly independent. Let M, N be the total number of possible combinations of each row and entire matrix respectively.
\begin{align}
    M &= 2^{n}
    \\ N &= 2^{n^2}
\end{align}
Thus,
\begin{enumerate}
    \item For the first row, number of combinations is $\brak{2^n-1}$ where 1 is the case where all elements are 0.
    \item For the second row, number of combinations is $\brak{2^n-1} - 1 = \brak{2^n-2^1}$. Another one is subtracted because we cannot count the vector that has been already used in first row.
    \item For the third row, number of combinations is $\brak{2^n-1} - 2-1 = \brak{2^n-2^2}$ because we have to omit 2 vectors from the count that already have been used in first and second row. And we have to omit one more vector that can be the linear combination of the first and second rows.
\end{enumerate}
And so on $\hdots$. Total number of non-singular matrices is
\begin{align}
    =\brak{2^n-2^0}\brak{2^n-2^1}\brak{2^n-2^2}\hdots\brak{2^n-2^{n-1}}
\end{align}
Thus,
\begin{multline}
    \Pr{(Y=0)} + \Pr{(Y=2)} = \\\frac{\brak{2^n-2^0}\brak{2^n-2^1}\brak{2^n-2^2}\hdots\brak{2^n-2^{n-1}}}{N}
\end{multline}
\begin{multline}
     \Pr{(Y=0)} + \Pr{(Y=2)} = 
     \\\frac{\brak{2^n-2^0}\brak{2^n-2^1}\brak{2^n-2^2}\hdots\brak{2^n-2^{n-1}}}{2^{n^2}} \label{equation 1}
\end{multline}
Since interchanging two rows multiplies the determinant by -1, the number of matrices with positive determinant is same as the number of matrices with negative determinant.
\begin{align}
    \Pr{(Y=0)} = \Pr{(Y=2)}  \label{equation 2}
\end{align}
Using equation \eqref{equation 2} in equation \eqref{equation 1}
\begin{multline}
     \Pr{(Y=2)} + \Pr{(Y=2)} = 
     \\\frac{\brak{2^n-2^0}\brak{2^n-2^1}\brak{2^n-2^2}\hdots\brak{2^n-2^{n-1}}}{2^{n^2}}
\end{multline}
\begin{multline}
     \Pr{(Y=2)} = 
     \\\frac{\brak{2^n-2^0}\brak{2^n-2^1}\brak{2^n-2^2}\hdots\brak{2^n-2^{n-1}}}{2^{n^2+1}}
\end{multline}
\end{document}
